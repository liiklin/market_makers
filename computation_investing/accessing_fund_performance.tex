\documentclass{article}
\begin{document}
    \centerline{\Large Metrics for Assessing Fund Performance}
    \section{Return}
    \begin{equation}
    return = \frac{price[end] }{price[begin]}- 1
    \end{equation}
    \section{Standard Deviation of Daily Return}
    
    \begin{equation}
        daily\_rets[i] = \frac{value[i]}{value[i-1]} -1 
    \end{equation} 
    
    \begin{equation}
        std\_metric = std(daily\_rets)
    \end{equation}
    \section{CAPM}
    Expected Returns of an asset depends on the expected return of the market, how the asset is normally responds to the market (known as beta $\beta$) and some asset specific knowledge (alpha $\alpha$).  Since CAPM assumes $\alpha$ is unknowable, or that if it's known the efficient market hypothesis prevents all be the very fastest from taking advantage of it, it's simply zeroed out and trading is done based on the beta.  The quants assume that $\alpha$ can be inferred based on additional market knowledge in ways which can be leveraged.

    $$r_i=\beta_i*r_m+\alpha_i$$

\end{document}